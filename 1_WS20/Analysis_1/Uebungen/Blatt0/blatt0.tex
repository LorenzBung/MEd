% document setup
\documentclass[12pt,a4paper]{article}
\usepackage[utf8]{inputenc}
\usepackage[ngerman]{babel}

% maths
\usepackage{amsfonts}
\usepackage{amssymb}
\usepackage{amsmath}

% utility
\usepackage{xcolor}
\usepackage[colorlinks=false,linkbordercolor=red,urlbordercolor=red]{hyperref}
\usepackage[shortlabels]{enumitem}
\usepackage{tikz}

% useful commands
\newcommand{\qed}{\null\nobreak\hfill\square}

% title, author etc.
\title{Analysis I, Blatt 0}
\author{Lorenz Bung (Matr.-Nr. 5113060), Gruppe 11\\\href{mailto:lorenz.bung@students.uni-freiburg.de}{\texttt{lorenz.bung@students.uni-freiburg.de}}}
\date{\today}

% begin document
\begin{document}

\maketitle


\section*{Aufgabe 0}

\begin{figure}[ht!]
    \centering
    \begin{minipage}[t][][b]{0.45\textwidth}
        \def\vennA{(0,0) circle (1.5cm)}
        \def\vennB{(60:2cm) circle (1.5cm)}
        \def\vennC{(0:2cm) circle (1.5cm)}
        \begin{tikzpicture}
        \draw \vennA node[below] {$A$};
        \draw \vennB node[above] {$B$};
        \draw \vennC node[below] {$C$};
        \draw (1.75,1) -- (3,2) node[right,fill=red!30]{$B \cap C$};

        \begin{scope}
            \clip \vennB;
            \fill[red,opacity=0.3] \vennC;
        \end{scope}
        \fill[blue,opacity=0.3] \vennA;
        \end{tikzpicture}
    \end{minipage}
    \begin{minipage}[t][][b]{0.45\textwidth}
        \def\vennA{(0,0) circle (1.5cm)}
        \def\vennB{(60:2cm) circle (1.5cm)}
        \def\vennC{(0:2cm) circle (1.5cm)}
        \begin{tikzpicture}
        \draw \vennA node[below] {$A$};
        \draw \vennB node[above] {$B$};
        \draw \vennC node[below] {$C$};
        \draw (0.25,1) -- (-1,2) node[left,fill=red!30]{$A \cup B$};
        \draw (1,-0.5) -- (1,-1.75) node[below,fill=blue!30]{$A \cup C$};

        \begin{scope}[even odd rule]
            \clip \vennB (-3,-3) rectangle (3,3);
            \fill[red,opacity=0.3] \vennA;
        \end{scope}
        \fill[red,opacity=0.3] \vennB;
        \fill[blue,opacity=0.3] \vennA;
        \begin{scope}[even odd rule]
            \clip \vennA (-3,-3) rectangle (5,3);
            \fill[blue,opacity=0.3] \vennC;
        \end{scope}
        \end{tikzpicture}
    \end{minipage}
\caption{$A \cup (B \cap C)$}
\end{figure}
\pagebreak


\section*{Aufgabe 1}

\begin{enumerate}[(i)]
    \item Die Potenzmenge $\mathcal{P}(M)$ enthält alle Teilmengen der Menge $M$. \begin{enumerate}[a)]
        \item $\mathcal{P}(\{1\}) = \{\emptyset, \{1\}\}.$
        \item $\mathcal{P}(\{1, 2, 3\}) = \{\emptyset, \{1\}, \{2\}, \{3\}, \{1, 2\}, \{1, 3\}, \{2, 3\}, \{1, 2, 3\}\}.$
        \item $\mathcal{P}(\emptyset) = \{\emptyset\}.$
    \end{enumerate}

    \item \begin{enumerate}[a)]
        \item falsch
        \item wahr
        \item falsch
        \item falsch
    \end{enumerate}

    \item $X = Y$, wenn $(X \subset Y) \wedge (Y \subset X).$ Die leere Menge ist (lt. Skript) jedoch Teilmenge jeder anderen Menge, somit auch der leeren Menge selbst. Daher gilt $X \subset Y$ und $Y \subset X$ und es folgt $X = Y$.

    \item Für Teilmengen gilt
    \begin{equation}
    \label{a1eq1}
    A \subset B \Leftrightarrow \forall x \in A: x \in B.
    \end{equation}
    Weiterhin gilt $A \cup B \Leftrightarrow x \in A \vee x \in B$ und $A \cap B \Leftrightarrow x \in A \wedge x \in B.$
    Aufgrund von (\ref*{a1eq1}) ist somit $A \cup B \overset{(\ref{a1eq1})}{\Leftrightarrow} x \in B \vee x \in B \Leftrightarrow x \in B$ (Schnitt analog).

    \item $(X \times Y) \cup (A \times Y) \Leftrightarrow \{(a, b) | a \in X, b \in Y\} \cup \{(c, d) | c \in A, d \in Y\}$\\
    $\Leftrightarrow x \in \{(a, b) | a \in X, b \in Y\} \vee x \in \{(c, d) | c \in A, d \in Y\}$
    %$\Leftrightarrow \{(a, b) | a \in X \wedge a \in A, b \in Y\} \Leftrightarrow $
\end{enumerate}


\section*{Aufgabe 2}

\begin{enumerate}[(i)]
    \item \begin{enumerate}
        \item $\{(a,b), (a,c), (b,c)\}$: Nein
        \item $\{(a,b), (b,a), (c,c)\}$: Ja, und zwar
        \[
        f: \left\{\begin{array}{l}
        a \mapsto b\\
        b \mapsto a\\
        c \mapsto c
        \end{array}\right..
        \]
    \end{enumerate}

    \item $A \subset B \Leftrightarrow \forall x \in A: x \in B \Rightarrow \forall y \in f(A): y \in f(B) \Rightarrow f(A) \subset f(B).$

    \item

    \item
\end{enumerate}


\section*{Aufgabe 3}

\begin{enumerate}[(i)]
    \item \begin{itemize}
        \item \textit{reflexiv}: Nein, da eine Gerade nicht zu sich selbst orthogonal sein kann (also $g \not\perp g$).
        \item \textit{symmetrisch}: Ja, denn $g \perp h \Leftrightarrow h \perp g$.
        \item \textit{antisymmetrisch}: Nein, da (im $\mathbb{R}^2$) $g \perp h \wedge h \perp i \Rightarrow g = i$.
        \item \textit{transitiv}: Nein, beispielsweise sei $g = i$ und $g \perp h$ sowie $h \perp i$. Dann ist jedoch nicht $g \perp i$, da $g = i$.
    \end{itemize}

    \item $R$ ist Äquivalenzrelation genau dann, wenn $R$ \textit{reflexiv}, \textit{symmetrisch} und \textit{transitiv} ist.
    \begin{itemize}
        \item \textit{reflexiv}: Ja, denn $g \parallel g$ (folgt aus Aufgabenstellung)
        \item \textit{symmetrisch}: Ja, denn $g \parallel h \Rightarrow h \parallel g$
        \item \textit{transitiv}: Ja, denn $g \parallel h \wedge h \parallel i \Rightarrow g \parallel i$.
    \end{itemize}
    Somit handelt es sich um eine Äquivalenzrelation.

    \item $\sim$ ist Äquivalenzrelation genau dann, wenn $\sim$ \textit{reflexiv}, \textit{symmetrisch} und \textit{transitiv} ist.
    \begin{itemize}
        \item \textit{reflexiv}: Ja, da $(r_1, r_2) \sim (r_1, r_2) \Leftrightarrow r_1r_2 = r_1r_2.$
        \item \textit{symmetrisch}: Ja: $(r_1, r_2) \sim (s_1, s_2) \Leftrightarrow r_1s_2 = r_2s_1 \Leftrightarrow s_2r_1 = s_1r_2 \Leftrightarrow s_1r_2 = s_2r_1 \Leftrightarrow (s_1, s_2) \sim (r_1, r_2).$
        \item \textit{transitiv}:
        \begin{align*}
        &(r_1, r_2) \sim (s_1, s_2) \wedge (s_1, s_2) \sim (t_1, t_2)\\
        \Leftrightarrow &r_1s_2 = r_2s_1 \wedge s_1t_2 = s_2t_1\\
        \Leftrightarrow &s_1 = \frac{r_1s_2}{r_2} \wedge s_1 = \frac{s_2t_1}{t_2}\\
        \Leftrightarrow &\frac{r_1s_2}{r_2} = \frac{t_1s_2}{t_2}\\
        \Leftrightarrow &r_1s_2t_2 = r_2s_2t_1\\
        \Leftrightarrow &r_1t_2 = r_2t_1\\
        \Leftrightarrow &(r_1, r_2) \sim (t_1, t_2).
        \end{align*}
        Somit ist $\sim$ eine Äquivalenzrelation.
    \end{itemize}
\end{enumerate}


\section*{Aufgabe 4}


% end document
\end{document}