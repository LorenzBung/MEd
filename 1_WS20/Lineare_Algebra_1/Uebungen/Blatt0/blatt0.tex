% document setup
\documentclass[12pt,a4paper]{article}
\usepackage[utf8]{inputenc}
\usepackage[ngerman]{babel}

% maths
\usepackage{amsfonts}
\usepackage{amssymb}
\usepackage{amsmath}

% utility
\usepackage{xcolor}
\usepackage[colorlinks=false,linkbordercolor=red,urlbordercolor=red]{hyperref}
\usepackage[shortlabels]{enumitem}
\usepackage{tikz}

% useful commands
\newcommand{\qed}{\null\nobreak\hfill\square}

% title, author etc.
\title{Lineare Algebra I, Blatt 0}
\author{Lorenz Bung (Matr.-Nr. 5113060), Gruppe 4\\\href{mailto:lorenz.bung@students.uni-freiburg.de}{\texttt{lorenz.bung@students.uni-freiburg.de}}}
\date{\today}

% begin document
\begin{document}

\maketitle


\section*{Aufgabe 1}

Induktionsbehauptung (IB): $n! > 2^n, n \geq 4.$\\
Induktionsanfang (IA) ($n = 4$): $n! = 4! = 4 * 3 * 2 * 1 = 24 > 16 = 2^4.$\\
Induktionsschritt (IS): $$(n + 1)! = (n + 1) * n! \overset{\text{(IB)}}{>} (n + 1) * 2^n \overset{n \geq 4}{>} 2^2 * 2^n = 2^{n + 2} > 2^{n + 1}.$$\\
$\qed$


\section*{Aufgabe 2}

Induktionsbehauptung (IB): $\sum\limits_{k = 1}^n \frac{1}{k (k + 1)} = \frac{n}{n + 1}, n > 0.$\\
Induktionsanfang (IA) ($n = 1$): $\sum\limits_{k = 1}^1 \frac{1}{k (k + 1)} = \frac{1}{1 * 2} = \frac{1}{2}.$\\
Induktionsschritt (IS):
$$\sum\limits_{k = 1}^{n + 1} \frac{1}{k (k + 1)} = \sum\limits_{k = 1}^n \frac{1}{k (k + 1)} + \frac{1}{(n + 1) (n + 2)}$$
$$\overset{\text{(IB)}}{=} \frac{n}{n + 1} + \frac{1}{(n + 1)(n + 2)} = \frac{n(n + 2) + 1}{(n + 1)(n + 2)}$$
$$= \frac{n^2 + 2n + 1}{(n + 1)(n + 2)} = \frac{(n + 1)^2}{(n + 1)(n + 2)} = \frac{n + 1}{n + 2}.$$\\
$\qed$


\section*{Aufgabe 3}
\label{sec:a3}

Zunächst wird gezeigt, dass
\begin{equation}
\label{eq1}
\sum\limits_{k = 0}^n 2^k = 2^{n + 1} - 1.
\end{equation}

\subsection*{Beweis zu \ref{eq1}}
Induktionsbehauptung (IB): $\sum\limits_{k = 0}^n 2^k = 2^{n + 1} - 1.$\\
Induktionsanfang (IA) ($n = 0$): $\sum\limits_{k = 0}^0 2^k = 2^0 = 1 = 2 - 1 = 2^1 - 1$.\\
Induktionsschritt (IS): $$\sum\limits_{k = 0}^{n + 1} 2^k = \sum\limits_{k = 0}^n 2^k + 2^{n + 1} \overset{\text{(IB)}}{=} 2^{n + 1} - 1 + 2^{n + 1} = 2^{n + 2} - 1.$$\\
$\qed$

\subsection*{Beweis zu \nameref{sec:a3}}
Induktionsbehauptung (IB): $f(n) \leq 2^{2^n}$.\\
Induktionsanfang (IA) ($n = 0$): $f(0) = 2 \leq 2 = 2^1 = 2^{2^0}$.\\
Induktionsschritt (IS):\\
Nach dem Satz von Euklid existiert eine Primzahl $m := 1 + \prod\limits_{k = 0}^n f(n).$\\
Da nicht garantiert ist, dass dies die nächstgrößere Primzahl ist, gilt zwar keine Gleichheit, jedoch auf jeden Fall $$f(n + 1) \leq 1 + \prod\limits_{k = 0}^n f(n).$$\\
Aufgrund (IB) ist
$$f(n + 1) \leq 1 + \prod\limits_{k = 0}^n f(n) \leq 1 + \prod\limits_{k = 0}^n 2^{2^n}$$
$$= 1 + (2^{2^0} * 2^{2^1} * \dots * 2^{2^n}) = 1 + 2^{2^0 + 2^1 + \dots + 2^n} = 1 + 2^{\sum_{k=0}^{n} 2^k}$$
$$\overset{\text{(\ref{eq1})}}{=} 1 + 2^{2^{n + 1} - 1} = \frac{2}{2} + \frac{2^{2^{n + 1}}}{2} = \frac{2^{2^{n + 1} + 1}}{2} = 2^{2^{n + 1}}.$$
$\qed$


\section*{Aufgabe 4}

\begin{enumerate}[(a)]
    \item Ja, da keine der durch $(0, 0)$ und $(0, 1)$ gehenden Geraden auch durch $(1, 0)$ geht.
    \item Nein, da gegenüberliegende Punkte auf dem Einheitskreis in der selben Äquivalenzklasse liegen. Beispiel: $[(0, 1)]_E = [(0, -1)]_E$.
\end{enumerate}

% end document
\end{document}