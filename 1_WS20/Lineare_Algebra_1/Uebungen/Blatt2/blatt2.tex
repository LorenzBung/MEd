% document setup
\documentclass[12pt,a4paper]{article}
\usepackage[utf8]{inputenc}
\usepackage[ngerman]{babel}

% maths
\usepackage{amsfonts}
\usepackage{amssymb}
\usepackage{amsmath}

% utility
\usepackage{xcolor}
\usepackage[colorlinks=false,linkbordercolor=red,urlbordercolor=red]{hyperref}
\usepackage[shortlabels]{enumitem}
\usepackage{tikz}

% useful commands and formatting
\newcommand{\qed}{\null\nobreak\hfill\square}
\setlength{\parindent}{0em}
\setlength{\parskip}{6pt}

% title, author etc.
\title{Lineare Algebra I, Blatt 2}
\author{Gruppe 4\\
    Lorenz Bung (Matr.-Nr. 5113060)\\
    \href{mailto:lorenz.bung@students.uni-freiburg.de}{\texttt{lorenz.bung@students.uni-freiburg.de}}\\
    Tobias Remde (Matr.-Nr. 5100067)\\
    \href{mailto:tobias.remde@gmx.de}{\texttt{tobias.remde@gmx.de}}
}
\date{\today}

% begin document
\begin{document}

\maketitle


\section*{Aufgabe 1}

\begin{enumerate}[(a)]
    \item \textbf{Behauptung}: $g_1 \sim g_2 \Leftrightarrow \exists h \in G (hg_1h^{-1} = g_2)$ ist eine Äquivalenzrelation.\\

    \textbf{Beweis}: $\sim$ ist eine Äquivalenzrelation, wenn $\sim$ \textit{reflexiv}, \textit{symmetrisch} und \textit{transitiv} ist.

    \textit{Reflexivität}.\\
    z.z.: $g_1 \sim g_1.$\\
    Bew.: $g_1 \sim g_1 \Leftrightarrow \exists h \in G (hg_1h^{-1} = g_1).$
    Sei $h$ das neutrale Element von $G$.
    Dann ist $hg_1h^{-1} = eg_1e^{-1} = eg_1e = g_1.$

    \textit{Symmetrie}.\\
    z.z.: $g_1 \sim g_2 = g_2 \sim g_1$.\\
    Bew.: $g_1 \sim g_2 \Leftrightarrow \exists h \in G (hg_1h^{-1} = g_2).$
    Existiere also ein solches $h \in G$.
    Dann ist \begin{align*}
        hg_1h^{-1} &= g_2\\
        hg_1 &= g_2h\\
        g_1 &= h^{-1}g_2h.
    \end{align*}
    Wähle nun $i := h^{-1}$: Dann existiert ein $i \in G (ig_2i^{-1} = g_1)$ und somit $g_2 \sim g_1$.

    \textit{Transitivität}.\\
    z.z.: $g_1 \sim g_2 \wedge g_2 \sim g_3 \Rightarrow g_1 \sim g_3$.\\
    Bew: $g_1 \sim g_2 \wedge g_2 \sim g_3 \Leftrightarrow \exists h,i \in G: hg_1h^{-1} = g_2 \wedge ig_2i^{-1} = g_3.$
    Dann ist $ig_2i^{-1} = ihg_1h^{-1}i^{-1}$.
    Da $(G, \cdot)$ eine Gruppe ist (und somit abgeschlossen), ist auch $(i \cdot h) \in G$ und damit auch $(i \cdot h)^{-1} \in G$.
    Sei nun $l := i \cdot h$.
    Dann ist $g_1 \sim g_3 \Leftrightarrow \exists l \in G (lg_1l^{-1} = g_3)$.

    Somit ist $\sim$ reflexiv, symmetrisch und transitiv und damit eine Äquivalenzrelation.\\
    $\qed$

    \item \textbf{Behauptung}: $[e] = \{e\}$.\\

    \textbf{Beweis}: $e \sim x \Leftrightarrow \exists h \in G: heh^{-1} = x$.
    Dann ist $heh^{-1} = (he)h^{-1} = hh^{-1} = e$.
    Also muss $x=e$ sein und damit $[e] = \{e\}$.\\
    $\qed$

    \item \textbf{Behauptung}: $(hgh^{-1})^{-1}$ liegt in der Äquivalenzklasse von $g^{-1}$.\\

    \textbf{Beweis}: Aufgrund von \textit{Bemerkung 1.21} im Skript ist $(a * b)^{-1} = b^{-1} * a^{-1}$.
    Somit ist $(hgh^{-1})^{-1} = (gh^{-1})^{-1}h^{-1} = (h^{-1})^{-1}g^{-1}h^{-1} = hg^{-1}h^{-1}$, was nach Definition in der Äquivalenzklasse von $g^{-1}$ liegt.\\
    $\qed$\\

    \textbf{Behauptung}: $(hgh^{-1})^n$ liegt in der Äquivalenzklasse von $g^{-n}$.\\

    \textbf{Beweis}: $(hgh^{-1})^n = \prod\limits_{k=0}^n hgh^{-1} = hgh^{-1} * \dots * hgh^{-1}.$

    Da $hgh^{-1} \in [g^{-1}]_\sim$, liegt jeder Faktor des Produkts in der Äquivalenzklasse von $g^{-1}$.
    Somit liegt das Gesamtprodukt in der Äquivalenzklasse von $g^{-1} * g^{-1} * \dots * g^{-1}$, also in $[g^{-n}]$.\\
    $\qed$

    \item \textbf{Behauptung}: Sei $r \in \mathbb{R}^*$ und $z \in \mathbb{Z}$. Dann ist $[r] = \{r\}$ und $[z] = \{z\}$.\\

    \textbf{Beweis}: Seien $g_1, g_2 \in \mathbb{R}^*$.
    Dann ist $g_1 \sim g_2 \Leftrightarrow \exists h \in \mathbb{R}^*: hg_1h^{-1} = g_2$.
    Da $(\mathbb{R}^*, \cdot)$ abelsch ist, ist jedoch
    $g_2 = hg_1h^{-1} = hh^{-1}g_1 = eg_1 = g_1$.
    Somit ist $[g_1] = \{g_1\}$.

    Beweis für $(\mathbb{Z}, +)$ ist analog (abelsche Gruppe).\\
    $\qed$
\end{enumerate}


\section*{Aufgabe 2}

\textbf{Behauptung}: Jede injektive Abbildung $f: X \rightarrow X$ ist auch surjektiv.\\

\textbf{Beweis}: Wir führen den Beweis induktiv über die Kardinalität $\#X$.

\textit{Induktionsanfang} ($\#X = 0$): trivial, da in diesem Fall $X = \emptyset$.

\textit{Induktionsvoraussetzung}: Sei nun $f: X \rightarrow X$ injektiv und surjektiv für $\#X = n$.

\textit{Induktionsschritt} ($n \Rightarrow n+1$):
Sei $\#X = n+1$.
Da $f$ für eine Menge mit $n$ Elementen sowohl injektiv als auch surjektiv ist, kann jedes Element im Definitionsbereich nur auf exakt ein Element im Wertebereich abgebildet werden.
Hat $X$ nun ein Element mehr, bleibt für dieses Element im Definitionsbereich nur noch ein Element im Wertebereich übrig, da $f$ für $\#X = n$ ja bereits bijektiv ist.
Somit ist $f$ auch für $\#X=n+1$ bijektiv.\\
$\qed$


\section*{Aufgabe 3}

\textbf{Behauptung}: Die Äquivalenzklasse von $(1\ 2\ 3)$ ist die folgende Menge: $[(1\ 2\ 3)] = \{id, (1\ 2\ 3), (1\ 3\ 2)\}$.

\textbf{Beweis}: Ein Element $g$ ist in der Äquivalenzklasse von $(1\ 2\ 3)$, falls gilt:

$\exists h \in S_4: h \cdot (1\ 2\ 3) \cdot h^{-1} = g.$

Es muss also einen Zyklus $(a_1\ \dots\ a_n) \in S_4$ geben, sodass \[(a_1\ \dots\ a_n)(1\ 2\ 3)(a_n\ \dots\ a_1) = (1\ 2\ 3).\]

Für die abzählbar vielen Elemente in $S_4$ kann nun einzeln geprüft werden, ob dies der Fall ist.
Es ergibt sich, dass nur die Elemente $id$, $(1\ 2\ 3)$ und $(1\ 3\ 2)$ in $[(1\ 2\ 3)]$ liegen.\\
$\qed$\pagebreak


\section*{Aufgabe 4}

\begin{enumerate}[(a)]
    \item \textbf{Behauptung}: $\mathcal{M}_{2\times2}(R)$ ist ein nicht-kommutativer Ring.\\

    \textbf{Beweis}: Es handelt sich um einen Ring, wenn:
    \begin{itemize}
        \item $(R, +)$ eine abelsche Gruppe mit neutralem Element $0_R$ ist
        \item $(R, \cdot)$ ein Monoid mit neutralem Element $1_R$ ist
        \item die Distributivgesetze $a \cdot (b + c) = a \cdot b + a \cdot c$ und $(a + b) \cdot c = a \cdot c + b \cdot c$ gelten.
    \end{itemize}

    \textit{Assoziativität von} $(\mathcal{M}_{2\times2}(R), +)$.\\
    Beh.: $(\mathcal{M}_{2\times2}(R), +)$ ist assoziativ.\\
    Bew.: Seien $a,\dots,l \in R$.
    Dann ist
    \begin{align*}
    &\left(\begin{pmatrix}
    a & b\\ c & d
    \end{pmatrix} + \begin{pmatrix}
    e & f\\ g & h
    \end{pmatrix}\right) + \begin{pmatrix}
    i & j\\ k & l
    \end{pmatrix}\\
    = &\begin{pmatrix}
    a+e & b+f\\ c+g & d+h
    \end{pmatrix} + \begin{pmatrix}
    i & j\\ k & l
    \end{pmatrix}\\
    = &\begin{pmatrix}
    a+e+i & b+f+j\\ c+g+k & d+h+l
    \end{pmatrix}\\
    = &\begin{pmatrix}
    a & b\\ c & d
    \end{pmatrix} + \begin{pmatrix}
    e+i & f+j\\ g+k & h+l
    \end{pmatrix}\\
    = &\begin{pmatrix}
    a & b\\ c & d
    \end{pmatrix} + \left(\begin{pmatrix}
    e & f\\ g & h
    \end{pmatrix} + \begin{pmatrix}
    i & j\\ k & l
    \end{pmatrix}\right).
    \end{align*}

    \textit{Kommutativität von} $(\mathcal{M}_{2\times2}(R), +)$.\\
    Beh.: $(\mathcal{M}_{2\times2}(R), +)$ ist kommutativ.\\
    Bew.: Seien $a,\dots,h \in R$.
    Dann ist
    \begin{align*}
        &\begin{pmatrix}
        a & b\\ c & d
        \end{pmatrix} + \begin{pmatrix}
        e & f\\ g & h
        \end{pmatrix}\\
        = &\begin{pmatrix}
        a + e & b + f\\ c + g & d + h
        \end{pmatrix}\\
        = &\begin{pmatrix}
        e+a & f+b\\ g+c & h+d
        \end{pmatrix}\\
        = &\begin{pmatrix}
        e & f\\ g & h
        \end{pmatrix} + \begin{pmatrix}
        a & b\\ c & d
        \end{pmatrix}.
    \end{align*}

    \textit{Neutrales von} $(\mathcal{M}_{2\times2}(R), +)$.\\
    Beh.: Die Matrix $0_{\mathcal{M}_{2\times2}(R)} := \begin{pmatrix}
    0&0\\0&0
    \end{pmatrix}$ ist das neutrale Element von $(\mathcal{M}_{2\times2}(R), +)$.\\
    Bew.: Seien $a,\dots,d \in R$.
    Dann ist
    \begin{align*}
        &\begin{pmatrix}
        a & b\\ c & d
        \end{pmatrix} + \begin{pmatrix}
        0&0\\0&0
        \end{pmatrix}\\
        = &\begin{pmatrix}
        a+0&b+0\\c+0&d+0
        \end{pmatrix}\\
        = &\begin{pmatrix}
        a&b\\c&d
        \end{pmatrix}.
    \end{align*}

    \textit{Inverses von} $(\mathcal{M}_{2\times2}(R), +)$.\\
    Beh.: Die Matrix $\begin{pmatrix}
    a & b\\ c & d
    \end{pmatrix} \in \mathcal{M}_{2\times2}(R)$ mit $a,\dots,d \in R$ hat das Inverse $\begin{pmatrix}
    -a & -b\\ -c & -d
    \end{pmatrix}$.\\
    Bew.: Seien also $a,\dots,d \in R$.
    Dann ist
    \begin{align*}
        &\begin{pmatrix}
        a & b\\ c & d
        \end{pmatrix} + \begin{pmatrix}
        -a & -b\\ -c & -d
        \end{pmatrix}\\
        = &\begin{pmatrix}
        a-a & b-b\\ c-c & d-d
        \end{pmatrix}\\
        = &\begin{pmatrix}
        0&0\\0&0
        \end{pmatrix} = 0_R.
    \end{align*}

    \textit{Assoziativität von} $(\mathcal{M}_{2\times2}(R), \cdot)$.\\
    Beh.: $(\mathcal{M}_{2\times2}(R), \cdot)$ ist assoziativ.\\
    Bew.: Seien $a,\dots,l \in R$.
    Dann ist
    \begin{align*}
        &\left(\begin{pmatrix}
        a&b\\c&d
        \end{pmatrix} \cdot \begin{pmatrix}
        e&f\\g&h
        \end{pmatrix}\right) \cdot \begin{pmatrix}
        i&j\\k&l
        \end{pmatrix}\\
        = &\begin{pmatrix}
        ae + bg & af + bh\\
        ce + dg & cf + dh
        \end{pmatrix} \cdot \begin{pmatrix}
        i&j\\k&l
        \end{pmatrix}\\
        = &\begin{pmatrix}
        aei+bgi+afk+bhk & aej+bgj+afl+bhl\\
        cei+dgi+cfk+dhk & cej+dgj+cfl+dhl
        \end{pmatrix}\\
        = &\begin{pmatrix}
        a&b\\c&d
        \end{pmatrix} \cdot \begin{pmatrix}
        ei+fk & ej+fl\\
        gi+hk & gj+hl
        \end{pmatrix}\\
        = &\begin{pmatrix}
        a&b\\c&d
        \end{pmatrix} \cdot \left(\begin{pmatrix}
        e&f\\g&h
        \end{pmatrix} \cdot \begin{pmatrix}
        i&j\\k&l
        \end{pmatrix}\right).
    \end{align*}

    \textit{Neutrales von} $(\mathcal{M}_{2\times2}(R), \cdot)$.\\
    Beh.: Die Matrix $1_{\mathcal{M}_{2\times2}(R)} := \begin{pmatrix}
    1&0\\0&1
    \end{pmatrix}$ ist das neutrale Element von $(\mathcal{M}_{2\times2}(R), \cdot)$.\\
    Bew.: Seien $a,\dots,d \in R$.
    Dann ist
    \begin{align*}
        &\begin{pmatrix}
        a&b\\c&d
        \end{pmatrix} \cdot \begin{pmatrix}
        1&0\\0&1
        \end{pmatrix}\\
        = &\begin{pmatrix}
        1a+0b&0a+1b\\
        1c+0d&0c+1d
        \end{pmatrix}\\
        = &\begin{pmatrix}
        a&b\\c&d
        \end{pmatrix}.
    \end{align*}

    Somit handelt es sich um einen Ring.

    Die Kommutativität des Monoids $(\mathcal{M}_{2\times2}(R), \cdot)$ (und damit auch des Ringes) ist jedoch nicht erfüllt:
    Seien $a,\dots,h \in R$.
    Dann ist
    \begin{align*}
        &\begin{pmatrix}
        a&b\\c&d
        \end{pmatrix} \cdot \begin{pmatrix}
        e&f\\g&h
        \end{pmatrix}\\
        = &\begin{pmatrix}
        ae + bg & af + bh\\
        ce + dg & cf + dh
        \end{pmatrix}\\
        \neq &\begin{pmatrix}
        ea + fc & eb + fd\\
        ga + hc & gb + hd
        \end{pmatrix}\\
        = &\begin{pmatrix}
        e&f\\g&h
        \end{pmatrix} \cdot \begin{pmatrix}
        a&b\\c&d
        \end{pmatrix}.
    \end{align*}\\
    $\qed$\pagebreak

    \item \textbf{Behauptung}: Wenn $R$ positive Charakteristik $k>0$ hat, ist die Charakteristik $char\ \mathcal{M}_{2\times2}(R) = k$.\\
    \textbf{Beweis}: \begin{align*}
        0_{\mathcal{M}_{2\times2}(R)} = &\begin{pmatrix}
        0_R&0_R\\0_R&0_R
        \end{pmatrix}\\
        = &\begin{pmatrix}
        \sum_{i=0}^{k}1_R & 0_R\\
        0_R & \sum_{i=0}^{k}1_R
        \end{pmatrix}\\
        = &\begin{pmatrix}
        \sum_{i=0}^{k}1_R & \sum_{i=0}^{k}0_R\\
        \sum_{i=0}^{k}0_R & \sum_{i=0}^{k}1_R
        \end{pmatrix}\\
        = &\sum\limits_{i=0}^k\begin{pmatrix}
        1_R & 0_R\\
        0_R & 1_R
        \end{pmatrix} = (1_{\mathcal{M}_{2\times2}(R)})^k.
    \end{align*}
    $\qed$
\end{enumerate}


% end document
\end{document}